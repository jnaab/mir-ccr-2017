%**************************************************************************
\section{Parallel Group Work}\label{sec:parallel-group-work}
%**************************************************************************

The afternoon sessions were used to discuss certain topics in more depth in
smaller groups. This section summarises the discussions of each group.

% ------------- Andreas Blenk (TUM LKN)
\subsection{SDN/NFV Measurements}

% ------------- Georg Carle
\subsection{SDN++: Applications Perspective}

The breakout session entitled SDN++ dealt with SDN from the perspective of how
to apply SDN, and how to introduce improvements to SDN (thereby creating
SDN++), for better meeting the identified requirements.  Participants of the
breakout session were Laurent Vanbevier, Artur Hecker, Wolfgang Kellerer,
Edwin Cordeiro and Georg Carle, the latter also being the presenter of the
results.  The method of the working group was first to identify relevant
application areas of SDN, then assess to which extent known SDN approaches
have shortcomings (i.e., identifying the `SDN pain areas`), and subsequently
identifying promising approaches for improving SDN.  The application areas of
SDN were (1) establishing means for programmability of the network, which can
be used for improving certain network properties, (2) management of advanced
cellular networks, in particular 5G networks, for different capabilities such
as network slicing, and (3) providing means to add sophisticated control
functionality to corporate networks, such as adding flexible access control.
Identified weaknesses of existing SDN were the fact that existing SDN
southbound interfaces, in particular OpenFlow, operate on a low level of
abstraction, which makes programming of the network time-consuming and error
prone.  Identified areas of improvement and need for further work were
specifying suitable high-level interfaces and abstractions.  There further is
the need to develop tools that are capable of automatically translate
high-level specifications to low-level configuration. A complete tool chain is
required.  This includes measurement tools that are capable of monitoring
changes. Network programmability is beneficial for measurement tools.  It is
expected that SDN management tools will facilitate to deal with the
programmability of networks.  Furthermore, verification tools will allow to
detect and prevent attempts of wrongly programming the network.    These tools
will form a network operating system, with tools that operate on top of the
operating system functions.  Another need for improvement is the development
of a clear transition path from today's networks to future SDN-based networks.
This includes to identify which legacy functionalities from today's networks
we assume being able to depend on in SDN deployments.



% ------------- Lars Eggert
\subsection{QUIC}

QUIC is a new UDP-based reliable transport protocol with build-in security and
optimized for HTTP/2 that is now under standardization by the IETF. QUIC was
originally proposed by Google and has already seen large-scale deployment for
Google services and in Google Chrome. Since September 2016 a new IETF working
group reviews the design of QUIC in order to publish a QUIC protocol
specification \cite{draft-ietf-quic-transport} with IETF consensus.

% Do you need to be Google or a large CDN deploy a protocol. Repeated failure
% of good ideas (HIP and SCTP). Only if two endpoints can do something and not
% require middleware to make changes is easier to deploy. Brian mentioned that
% QUIC works because controls both endpoints (browser and servers) and also
% pointed to TSV area slides.

% Friday notes from Mirja:
% -wire image: what will be visible to the network for managibilty
% - do you need to be a mayor internet player to deploy something? good ideas
%   and good engineering needs also the right incentives to see deployment.
%   particle deployability with jus one player already needs benefits. failure
%   of hip, sctp, ? quic only need to change both endpoints (browser, and
%   server) but still only big organizations can do that. work on transition
%   strategies in IAB (draft). if deployment incentives are not aligned, it
%   will not fly no better how good it is. First adopter problem: if no-one
%   has incentives/benefits to go first, hard to convince people to invest
%   even if the final approach would provide better results. In-network
%   support needs also benefits if only one network adapts it, e.g. in data
%   centers or local optimization.
% - measurement provided by quic. Are additional measurements needed? Only few
%   can do this large-scale measurements. How valuable are measurements of
%   smaller networks (non-google).


% ------------- Johannes Naab / Heiko Niedermayer
\subsection{DDoS Defence beyond Centralization}

% ------------- Alexander von Gernler
\subsection{Security}

The security breakout session covered civil liberties and privacy.

Firstly, the group set its focus and decided not to discuss the topics of
trustworthy hardware or civil liberties, but instead to concentrate on SDN
security and problems of cloudification

Key results: 1) Customer networks are converging:  Customers want less own
hardware, and want to be more independent and to lease remote services and
equipment rather than owning it.  2) Virtualization (which happens when you
cloudify applications) amplifies known problems in traditional fields like
security, trust, verifiability or visibility.  3) A special challenge is the
cloudification of services that already utilize virtualization in the
traditional model, for example sandboxes that analyze malware.  For a cloud
case, one would end up with nested virtualization, which in turn comes with
even new problems concerning performance and visibility of the virtualization
to the malware being inspected 4) Encryption of data still leads to the
usability of cloud scenarios being reduced to mostly SaaS, because homomorphic
encryption is still not there to solve these problems 5) Special problems with
end-to-end security, e.g., there is more end-to-end encryption happening,
which is good.  As a downside however, it makes life harder for people
inspecting traffic in the middle If termination of encrypted connections is
done in the cloud, there will be an unencrypted last mile as new security
issue arising from this scenario.

% ------------- Aaron
\subsection{IoT}
