%**************************************************************************
\section{Parallel Group Work}\label{sec:parallel-group-work}
%**************************************************************************

The afternoon sessions were used to discuss certain topics in more depth in
smaller groups. This section summarises the discussions of each group.

% ------------- Andreas Blenk (TUM LKN)
\subsection{SDN/NFV Measurements}

% ------------- Georg Carle
\subsection{SDN++: Applications Perspective}

The breakout session entitled SDN++ dealt with SDN from the perspective of how to apply SDN, and how to introduce improvements to SDN (thereby creating SDN++), for better meeting the identified requirements.
Participants of the breakout session were Laurent Vanbevier, Artur Hecker, Wolfgang Kellerer, Edwin Cordeiro and Georg Carle, the latter also being the presenter of the results.
The method of the working group was first to identify relevant application areas of SDN, then assess to which extent known SDN approaches have shortcomings (i.e., identifying the `SDN pain areas`), and subsequently identifying promising approaches for improving SDN.
The application areas of SDN were (1) establishing means for programmability of the network, which can be used for improving certain network properties, (2) management of advanced cellular networks, in particular 5G networks, for different capabilities such as network slicing, and (3) providing means to add sophisticated control functionality to corporate networks, such as adding flexible access control. 
Identified weaknesses of existing SDN were the fact that existing SDN southbound interfaces, in particular OpenFlow, operate on a low level of abstraction, which makes programming of the network time-consuming and error prone. 
Identified areas of improvement and need for further work were specifying suitable high-level interfaces and abstractions.
There further is the need to develop tools that are capable of automatically translate high-level specifications to low-level configuration. A complete tool chain is required.
This includes measurement tools that are capable of monitoring changes. Network programmability is beneficial for measurement tools.
It is expected that SDN management tools will facilitate to deal with the programmability of networks.
Furthermore, verification tools will allow to detect and prevent attempts of wrongly programming the network.    
These tools will form a network operating system, with tools that operate on top of the operating system functions.
Another need for improvement is the development of a clear transition path from today's networks to future SDN-based networks. This includes to identify which legacy functionalities from today's networks we assume being able to depend on in SDN deployments.
           



% ------------- Lars Eggert
\subsection{QUIC}

% Do you need to be Google or a large CDN deploy a protocol. Repeated failure of good ideas (HIP and SCTP). Only if two endpoints can do something and not require middleware to make changes is easier to deploy. Brian mentioned that QUIC works because controls both endpoints (browser and servers) and also pointed to TSV area slides.


% ------------- Johannes Naab / Heiko Niedermayer
\subsection{DDoS Defence beyond Centralization}

% ------------- Alexander von Gernler
\subsection{Security}

% ------------- Aaron
\subsection{IoT}