%**************************************************************************
\section{Invited Presentations}\label{sec:invited-presentations}
%**************************************************************************

The invited presentations were intended as a basis for triggering discussions
and identifying areas for group work.

% ------------ Dirk Kutscher
\subsection{Edge Computing Considered Harmful}
\todo[inline]{Dirk Kutscher}


% ------------ Alexander von Gemler
%\subsection{Towards A Clean Slate -- Digital Sovereignty in the Post Snowden Era}
\subsection{Digital Sovereignty in the Post Snowden Era}

In this talk, Alexander von Gemler (genua GmbH) emphasizes the need for the
availability of trustworthy hardware and trustworthy operating systems for the
common user.  The postulation is that without these means, democracy itself
will suffer in the long run, as people need not only to be unwatched and
uncensored when pursuing their forming and expressing of political opinion,
but they also need to feel unwatched and uncensored.  If these prerequisites
are not met, chilling effects will occur, and the users will adjust their
behaviour to whatever they think is socially appropriate.  The talk finishes
by enumerating some possible ways out of the situation, and appeals to the
conscience of the computer scientists around, as they are needed to explain
the problem to society, and ultimately solve it.


% ------------ Artur Hecker
\subsection{On software network management}

In this talk, Artur Hecker (Huawei) argues that the paradigm change brought by
software networks does not suit well any planning approach for network
dimensioning and design, including but not limited to planning or
pre-provisioning of management and control planes per se. As an example, OF
\ac{SDN} and ETSI \ac{NFV} silently rely on pre-established, fixed control
networks; opening these up for programmability currently bears risks with
respect to the integrity of the control plane easily leading to e.g. a
self-lockout. To overcome this, he proposes a new model and a new protocol, a
sort of a least common denominator for software networks, whose only
well-defined purpose it is to autonomously construct, adjust and maintain
control plane including the placement of control compute nodes and control
paths without presuming any particular network purpose.

% ------------ Wolfgang Kellerer
%\subsection{FlexNets: Quantifying Flexibility in Communication Networks}
%\subsection{Quantifying Flexibility in Communication Networks}
\subsection{FlexNets: It's all about Flexibility}

New requirements for communication networks include the need for dynamic
changes of the required networking resources. Providing the required
flexibility to react to those changes and being cost efficient at the same
time has recently emerged as a huge challenge in networking research. With
\ac{SDN} and \ac{NFV}, three concepts have emerged in the networking research,
which claim to provide more flexibility. However a deeper understanding of the
flexibility vs. cost trade-off is missing so far in networking research. In
this talk, Wolfgang Kellerer (TUM) proposes a definition for flexibility as a
new measure for network design space analysis \cite{wkellerer:infocom:2016}
and gives an illustrative example with \ac{SDN} controller placement.

% ------------ Brian Trammell
\subsection{An Accidental Internet Architecture}

The Internet, as seen from the point of view of the applications, services,
and user agents it connects, is defined by the interfaces it provides. Brian
Trammell (ETH Zürich) in this talk, introduced PostSockets
\cite{draft-trammell-post-sockets}, a work-in-progress proposal to reimagine
the Internet from a new API down. PostSockets provides for secure,
message-oriented, explicitly multipath, asynchonous communication.  It
separates long-term state (cryptographic identity and resumption parameters)
from ephemeral per-path state (transport connection windows, session secrets),
and is suitable both for reliable message stream transports (such as QUIC for
HTTP/2) as well as for partially-reliable media applications. PostSockets is
intended to allow applications to be developed separate from (possibly
runtime-bound) transport protocol dynamics, in turn accelerating the
deployment of recent innovations at Layer 4

% ------------ Vaibhav Bajpai
\subsection{Measuring IPv6 Performance}

A large focus of IPv6 measurement studies in the past has been on measuring
IPv6 adoption on the Internet. This involved measuring addressing, naming,
routing and reachability aspects of IPv6.  However, there has been very little
to no study on measuring IPv6 performance. Vaibhav Bajpai (Jacobs University
Bremen) shows that his dissertation work fills this gap. He uses 80
dual-stacked SamKnows \cite{vbajpai:comst:2015} probes deployed at the edge of
the network to measure IPv6 performance of operational dual-stacked content
services on the Internet.  He presents a comparison of how content delivery
\cite{vbajpai:networking:2015, sahsan:pam:2015} over IPv6 compares to that of
IPv4. He shows how in the process, he also identified glitches in this content
delivery \cite{seravuchira:cnsm:2016} that once fixed can help improve user
experience over IPv6. His also points out areas of improvements
\cite{vbajpai:anrw:2016} in the standards work for the IPv6 operations
community at the IETF. This study can be relevant for network operators that
are either in the process of or are in early stages of IPv6 deployment.


% ------------ Minoo Ruohi
%\subsection{Path tracing and validation of IPv4 and IPv6 siblings}
\subsection{Classification of IPv4-IPv6 Siblings}

With the growing deployment of IPv6, the question arises whether and to what
extent this new protocol is co-deployed with IPv4 on existing hardware or
whether new hardware or proxy solutions are deployed. Understanding the
resulting cross-dependencies between IPv4 and IPv6 hosts will add a
significant level of insight into Internet structure and resilience research.
In this talk, Minoo Ruohi (TUM) presented an active measurement technique to
determine whether an IPv4-IPv6 address pair resides on the same physical host.
The measurement technique is based on measuring clock skew through TCP
timestamps, and introduces new capabilities to classify nonlinear clock skews.
In their studies, they achieve 97.7\% accuracy on a ground truth data set of
458 hosts and have proved this technique's value by applying it to 371K
sibling candidates, of which they classify 80K as siblings. A technical report
on this work has been published~\cite{Scheitle2016}. Further, the classified
siblings as well as additional data and all code from this work have been
released for public use~\footnote{https://github.com/tumi8/siblings}.


% ------------ Laurent Vanbever
%\subsection{SWIFT: Predictive Fast Reroute upon Remote BGP Disruptions}
\subsection{SWIFT: Predictive Fast Reroute upon Remote BGP Disruptions}

Fast rerouting upon network failure is a key requirement when it comes to meet
stringent service-level agreements. While current frameworks enable sub-second
convergence upon local failures, they do not protect against the much frequent
remote failures. In contrast to local failures, learning about a remote
failure is fundamentally slower as it involves receiving potentially hundred
of thousands of BGP messages. Also, pre-populating backup forwarding rules is
impossible as any subset of the prefixes can be impacted.

In this presentation, Laurent Vanbever (ETH Zurich) presented SWIFT, a general
fast re-route framework supporting both local and remote failures~\cite{}.
SWIFT is based on two novel techniques. First, SWIFT copes with slow
notification by predicting the overall extent of a remote failure out of few
control-plane (BGP) messages. Second, SWIFT introduces a new data-plane
encoding scheme which enables it to quickly and flexibly update the impacted
forwarding entries.

SWIFT have been implemented by the ETH \ac{NSG} and its performance benefits
have been demonstrated by showing that: i) SWIFT is able to predict the extent
of a remote failure with high accuracy (?93\%); and ii) SWIFT encoding scheme
enables to fast-converge more than 95\% of the impacted forwarding entries.
Overall, SWIFT reduces the average convergence time from few minutes to few
seconds.

%\footnote{http://synet.ethz.ch/}

% ------------ Christian Prehofer
\subsection{Open Platforms for Cyber-physical systems}

For many cyber-physical systems, there is a strong trend towards open systems
which can be extended during operation by instantly adding functionalities on
demand. In this talk, Christian Prehofer (Fortiss) discussed this trend in the
context of networked systems in the automotive, medical and industrial
automation areas and elaborated the research challenges of platform for such
open, networked systems. A main problem is that such CPS apps shall be able to
access and modify safety critical device internals. Further, results of the
TAPPS (Trusted Apps for open CPS) project were presented, which  develops an
end-to-end solution for development and deployment of trusted
apps~\cite{Prehofer2016}. This includes trusted hardware and virtualization of
networking and CPU, as well as dedicated execution environments and
development support for trusted apps.


% ------------ Holger Kinkelin
\subsection{Collaborative intrusion handling using the Blackboard-Pattern}

Defending computer networks from ongoing security incidents is a key
requirement to ensure service continuity. Handling incidents is a complex
process consisting of the three steps: 1) intrusion detection, 2) alert
processing and 3) intrusion response. For useful and automated incident
handling a comprehensive view on the process and tightly interleaved single
steps are required. Existing solutions for incident handling merely focus on a
single step leaving the other steps completely aside. Incompatible and
encapsulated partial solutions are the consequence. In this talk Holger
Kinkelin (TUM) proposed an approach on incident handling based on a novel
execution model that allows interleaving and collaborative interaction between
the incident handling steps using the Blackboard Pattern. Their holistic
information model lays the foundation for a conflict free collaboration. The
incident handling steps are further segmented into exchangeable functional
blocks distributed across the network. To show the applicability of their
approach, typical use cases for incident handling systems were identified and
tested based on their implementation. This talk was based on a paper published
at WISCS associated to ACM CCS 2016~\cite{Herold2016}.
