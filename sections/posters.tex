%**************************************************************************
\section{Posters}\label{sec:posters}
%**************************************************************************

Participants were also encouraged to volunteer to bring a poster to provide a
perspective into their recent measurement research work.

%\subsection{The Cost of Security in the SDN Control Plane}
\subsection{Cost of Security in the SDN Control Plane}

In OpenFlow enabled \ac{SDN} network control is carried out remotely via a
control connection. In order to deploy OpenFlow in production networks,
security of the control connection is crucial. For OpenFlow connections, TLS
encryption is recommended by the specification. In this work
\cite{wkellerer:conext:2015}, we analyze the TLS support in the OpenFlow
eco-system. In particular, we implemented a performance measurement tool for
encrypted OpenFlow connections, as there is non available.  Our first results
show that security comes at an extra cost and hence further work is needed to
design efficient mechanisms taking the security-delay trade-off into account.

%Published: R. Durner, W. Kellerer, The cost of Security in the SDN control
%Plane, ACM CoNEXT 2015 - Student Workshop, Heidelberg, Germany, Dezember 2015.

%\subsection{THE BALTIKUM TESTBED - Selected Activities in the Baltikum Testbed}
\subsection{The Baltikum Testbed}

The poster showed a high-level overview to the recent activities in the
Baltikum Testbed. The testbed which is focussed on performance measurements of
x86-based packet processing systems provides an automated, documented, and
reproducible experiment workflow. The poster presented several activities,
comprising the load generator MoonGen, automated benchmarks of routers and
OpenFlow switches, and different performance studies, including an IPsec
gateway with NIC-offloading.

Ref: [1] P. Emmerich, S. Gallenmüller, D. Raumer, F. Wohlfart, and G. Carle.
MoonGen: A Scriptable High-Speed Packet Generator. In Internet
Measurement Conference 2015 (IMC’15), Tokyo,
Japan, October 2015.
[2] Sebastian Gallenmüller, Paul Emmerich, Florian Wohlfart, Daniel
Raumer, and Georg Carle. Comparison of Frameworks for High-Performance
Packet IO. In ACM/IEEE Symposium on
Architectures for Networking and Communications Systems (ANCS 2015),
Oakland, CA, USA, May 2015.
[3] Daniel Raumer, Sebastian Gallenmüller, Paul Emmerich, Lukas Märdian,
Florian Wohlfart, and Georg Carle. Efficient serving of VPN endpoints on
COTS server hardware. In IEEE 5th
International Conference on Cloud Networking (CloudNet’16), Pisa, Italy,
October 2016.
[4] Daniel Raumer, Sebastian Gallenmüller, Florian Wohlfart, Paul
Emmerich, Patrick Werneck, and Georg Carle. Revisiting benchmarking
methodology for interconnect devices. In Applied
Networking Research Workshop 2016 (ANRW ’16), Berlin, Germany, 2016.


\subsection{Boost Virtual Network Resource Allocation: Using Machine Learning for Optimization}

Rapidly and efficiently allocating virtual network resources, i.e.,
solving the online Virtual Network Embedding (VNE) problem is
important in particular for future communication networks. We
propose a system using an admission control to improve the performance
for the online VNE problem. The admission control implements a Neural
Network that classifies virtual network requests based on network
representations, which are using graph and network resource features
only. Via simulations, we demonstrate that the admission control,
i.e., the Neural Network filters virtual network requests that are
either infeasible or that need too long for being efficiently
processed. Thus, our admission control reduces the overall system
runtime, i.e., it improves the overall calculation efficiency for
the online VNE problem.
Generally, we demonstrate the ability to learn from the history
of VNE algorithms. We show that it is possible to learn the behavior
of algorithms and how to integrate this knowledge when solving future
problem instances.

Reference:
[1] A. Blenk, P. Kalmbach, P. van der Smagt, W. Kellerer, Boost Online
Virtual Network Embedding: Using Neural Networks for Admission Control,
12th International Conference on Network and Service Management (CNSM),
 Montreal, Quebec, Canada, Oktober 2016.

\subsection{HyperFlex: Towards Flexible, Reliable and Dynamic SDN Virtualization Layer}

The virtualization of Software Defined Networks (SDN) allows multiple tenants
to share a physical SDN infrastructure, where each tenant can bring its own
controller for a flexible control of its virtual SDN network (vSDN). In order
to virtualize SDN networks, a network hypervisor is deployed between the
physical infrastructure and the tenants' controllers. We present, HyperFlex, a
flexible, reliable and dynamic SDN virtualization layer. HyperFlex achieves
the flexibility of deploying hypervisor functions as software or alternatively
using available processing capabilities of network nodes. It also provides
resources isolation for the control plane of vSDNs. Additionally, HyperFlex
supports the dynamic migration of network hypervisor instances on run time.
These features are key steps towards vigorous slicing in 5G.

References:
[1] A. Blenk, A. Basta, M. Reisslein, and W. Kellerer, “Survey on Network Virtualization Hypervisors
for Software Defined Networking,” IEEE Communications Surveys \& Tutorials, pp. 1–32, 2015.
[2] A. Blenk, A. Basta, and W. Kellerer, “HyperFlex: An SDN virtualization architecture with flexible
hypervisor function allocation,” in Proc. IFIP/IEEE Conf. IM, pp. 397–405, 2015.
[3] A. Basta, A. Blenk, H. Belhaj Hassine, and W. Kellerer, “Towards a dynamic SDN virtualization
layer: Control path migration protocol,” in Proc. ManSDN/NFV Workshop (CNSM), 2015.
[4] A. Blenk, A. Basta, J. Zerwas, M. Reisslein, and W. Kellerer, “Control plane latency with sdn
network hypervisors: The cost of virtualization,” IEEE Transactions on Network and Service
Management, pp. 360–380, 2016.
[5] A. Basta, A. Blenk, Y.-T. Lai, and W. Kellerer, “HyperFlex: Demonstrating control-plane isolation
for virtual software-defined networks,” in Proc. IFIP/IEEE Conf. IM, pp. 1163–1164., 2015.

\subsection{SafeCloud}

The poster gives an overview of the cloud security activities of the
SafeCloud project. Safe cloud usage for the user requires privacy and in
SafeCloud a variety of privacy-enhanced services are developed. This
includes cryptographic databases and secure multiparty computation.
Security and resilience mechanisms add diverse and censorship-resistant
storage, multipath and route monitoring.

Reference:
safecloud-project.eu

\subsection{sKnock: Scalable Secure Port Knocking}

Port-knocking  is  the  concept  of  hiding  remote  services behind   a
  firewall   which   allows   access   to   the   services’ listening
ports only after the client has successfully authenticated to  the
firewall.  This  helps  in  preventing  scanners  from  learning what
services  are  currently  available  on  a  host  and  also  serves as
a  defense  against  zero-day  attacks.  Existing  port-knocking
implementations are not scalable in service provider deployments due to
their usage of shared secrets. Here, we introduce an  implementation  of
port-knocking  based  on  x509  certificates aimed  towards  being
highly  scalable.

Reference:
Daniel Sel, Sree Harsha Totakura, Georg Carle, “sKnock: Scalable
Port-Knocking for Masses ,” in Workshop on Mobility and Cloud Security \&
Privacy, Budapest, Hungary, Sep. 2016.

\subsection{BMBF Project SarDiNe}

The BMBF project SarDiNe is motivated by the advent of the virtualization of
complete enterprise networks. Software defined networks (SDN) tremendously ease
the creation and management of virtual networks which leads to new challenges
in security policy enforcement. Traditionally, networks were separated
physically and security was mainly enforced by firewalls placed at gateway
positions between the physical networks. With highly dynamic virtual networks
it remains unclear where to place firewalls, especially if higher security
measures like filtering on the application layer are needed.

In SarDiNe we propose to virtualize firewall functionality as well and
dynamically place it on commodity hardware managed by cloud techniques and
spread across the network. Then, the SDN is used to dynamically reroute traffic
via these virtual network functions (VNF). This approach promises a scalable
and cost-efficient security solution applyable in many different setups. As main
use case we elaborate a bring-your-own-device (BYOD) scenario. Also, we are
interested in exploiting the SDN to provide parts of the filtering
functionality in its fast switching hardware. The result is a hybrid VNF-SDN
firewall which aims at a cost reduction in terms of computation resources
needed for scaling and latency imposed by the rerouting.

\subsection{Securebox}
\todo[inline]{TBA}

\subsection{StackMap}
\todo[inline]{TBA}

\subsection{PATHspider: A tool for active measurement of path transparency}

In today’s Internet we see an increasing deployment of middleboxes. While 
middleboxes provide in-network functionality that is necessary to keep 
networks manageable and economically viable, any packet mangling – whether 
essential for the needed functionality or accidental as an unwanted side 
effect – makes it more and more difficult to deploy new protocols or 
extensions of existing protocols. For the evolution of the protocol stack, 
it is important to know which network impairments exist and potentially 
need to be worked around. While classical network measurement tools are 
often focused on absolute performance values, we present a new measurement 
tool, called PATHspider that performs A/B testing between two different 
protocols or different protocol extension to perform controlled experiments 
of protocol-dependent connectivity problems as well as differential treatment. 
PATHspider is a framework for performing and analyzing these measurements, 
while the actual A/B test can be easily customized. This paper describes the 
basic design approach and architecture of PATHspider and gives guidance how 
to use and customize it.

\subsection{FlexNets: Quantifying Flexibility in Communication Networks}

Communication networks have emerged to become the basic infrastructure for all areas of our society with application areas ranging from social media to industrial production and healthcare. New requirements include the need for dynamic changes of the required resources, for example, to react to social events or to shifts of demands. Existing networks and, in particular, the Internet cannot meet those requirements mainly due to their ossification and hence limitation in resource allocation, i.e., lack of flexibility to adapt the available resources to changes of demands on a small time-scale and in an efficient way. In recent years, several concepts have emerged in networking research to provide more flexibility in networks through virtualization and control plane programmability. In particular, the split between data plane and a centralized control plane as defined by Software Defined Networking (SDN) is regarded as the basic concept to allow flexibility in networks. However, a deeper understanding of what flexibility means remains open. In this project, flexibility focuses on the dynamic changes in time and size of a network that is characterized by its resources (link rate and node capacities) and connectivity (network graph). It is the objective of this research to analyse the fundamental design space for flexibility in SDN-based networks with respect to cost such as resource usage, traffic overhead and delay. The outcome will be a set of quantitative arguments pro and contra certain design choices. An analytical cost model to quantitatively assess the trade-off for flexibility vs. cost will be developed. To assess flexibility with respect to general graph properties a graph model will be designed. The detailed analysis is based on three use cases: dynamic resource allocation, QoS control, and resilience. In the state of the art, selected aspects of flexibility have been explored for certain network scenarios, a fundamental and comprehensive analysis is missing.
